\documentclass[a4wide,fontsize=12pt]{article}
\usepackage[utf8]{inputenc}
\usepackage{scrextend}
\usepackage{authblk}
\usepackage[russian,english]{babel}
%\usepackage[english]{babel}
\usepackage{amsmath}
\usepackage[left=2cm,right=2cm,
    top=2cm,bottom=2cm,bindingoffset=0cm]{geometry}
\usepackage{amsfonts}
\usepackage{bm}
\usepackage{graphicx}
\usepackage{hyperref}
\usepackage{comment}
\usepackage{tikz}
\usetikzlibrary{positioning}
\usetikzlibrary{patterns}
\usepackage{pgfplots}
\usepackage{multicol}
% \usepackage{filecontents}
\usepackage{animate}
\usepackage{graphics}

\usepackage{subcaption}
\usepackage{caption}
% \captionsetup[figure]{font=small,skip=1pt}     %% Adjust here or equivalently

\usepackage{tikz}
\usepackage{float}
\usetikzlibrary{decorations.pathreplacing}

\title{\textbf{Methods of visualisation for flows with internal waves attractors}}
\author[1,A]{D.A Ryazanov}
\author[2,B]{S.A. Elistratov}
\author[3,C]{I.N. Sibgatullin}
\author[4,A]{M.V.~Kraposhin}

\affil[A]{Ivannikov Institute for System Programming of the RAS}
\affil[B]{Lomonosov Moscow State University}
\affil[C]{Shirshov Institute of Oceanology of Russian Academy of Sciences}
\affil[ ]{}
\affil[1]{ORCID: 0000-0001-9568-7121}
\affil[2]{ORCID: 0000-0002-7006-6879}
\affil[3]{ORCID: 0000-0003-2265-3259}
\affil[4]{ORCID: 0000-0001-5730-2702}

%Ivannikov Institute for System Programming of the RAS, Lomonosov Moscow State University, Shirshov Institute of Oceanology of Russian Academy of Sciences.

\date{}

\begin{document}

\maketitle

\underline{\textbf{Abstract}}

Application of different approaches for visualization of hydrodynamic fields of internal waves is closely connected with the possibility of extracting important numerical characteristics of the flows. In this paper we describe methods of visualisation, which were shown to be very effective for the illustration of the internal wave attractors both for laboratory experiments and numerical simulations. On the other hand, some novel approaches for description of the wave flows with accumulation of energy are discussed. Methods of vortices identification show interesting properties of the wave flows, and may give an alternative for the estimation of the width of wave beams. At the same time, the limitations of the applicability of the described methods strongly depend on the space and time resolution of the available data, which is especially important in the areas of the internal wave beam reflection.  

\textbf{Keywords:} CFD, internal waves, inertial waves, post-processing, wave attractor, open-source software 

\section{Introduction}

Internal waves are may be the most common types of waves in oceans, since even if the ocean surface seems to be calm, the deep-ocean waves \cite{1966MunkAbyssalRecipes,Munk1998} are always present\footnote{``Gravity waves in the ocean's interior are as common as waves at the sea surface\,--\,perhaps even more so, for no one has ever reported an interior calm''~\cite{2005Munk9IW}}. Oceanologists, biologists, ecologists, and technician have specific  interests in the study of internal waves. This phenomenon is partly responsible for the vertical mixing of the stratified fluids, the migration of living organisms, the redistribution of energy in the ocean, and the propagation of various kinds of impurities and pollution.

Despite a great diversity of experimental approaches for visualisation of weakly compressible flows \cite{znamenskaya2021methods388562176,2014SutherlandDuaxoisPeacockIWinLE}, the experiments are often bounded by the possibility to assess some interesting propriety at certain time and location. Concerning the internal attractors and accompanying phenomena, remarkable advantages have been achieved since 199X. Besides the successful application of traditional approaches based on synthetic schlieren, particle image velocimetry (PIV) and planar laser-induced fluorescence (PLIF) , some methods based on signal analysis, which traditionally were more applied to electromagnetic fields, showed remarkable relevance for description of the internal or inertial wave dynamics~\cite{2013BourgetDauxoisJoubaudOdier,2016DossmannBourgetBrouzetDauxoisJoubaudOdier,2019DavisDauxoisJaminJoubaud,2020HusseiniVarmaDauxoisJoubaudOdierMathur}. Meanwhile, the time and space resolution of these methods is limited, and sometimes the laboratory data do not allow to reconstruct the hydrodynamic fields, as mentioned in~\cite{2014LeePaolettiSwinneyMorrisonExperDeterIWpower} concerning the reconstruction of the stream function based on laboratory data near the boundaries.

Internal wave appear as a result of perturbations of stably stratified fluid. 
One of the major sources of the external forcing, resulting in internal waves, is the tidal effect produced by the orbital motions of the Moon and the Sun. These global flows interact with the ocean bottom irregularities and generate internal waves. The very special feature of the internal waves is that in case of constant stratification the angle of the wave beams and the law of reflection are defined by the dispersion relation:
\begin{equation}
\frac{\omega_0}{N} = \sin(\theta),
\label{eq:DispRel}
\end{equation}
where $\omega_0$ is the forcing frequency determined by external forcing on the fluid; 
$N$~-- buoyancy frequency.

% Пояснительная картинка с распространением
% Фокусировка внутренних волн

Due to these properties internal waves can be focused upon reflection from the oblique wall. As a consequence,  the internal wave wavelength is reduced upon reflection but its amplitude increases.

% Аттракторы внутренних волн
% картинка с резервуаром

The great interest is the result of the consecutive reflections. It can be obtained by the monochromatic excitation of the internal waves in the closed domain with the oblique boundaries, the simplest example of which is the trapeze with one inclined boundary.
A remarkable property of the billiard in trapezoid geometry is that focusing prevails over defocusing.
If the internal waves beams propagate in trapezoid tank, focusing occurs continuously and all the wave beams converge to closed trajectory.  The simplest one is a parallelogram with four reflection points located at the sides of the trapeze (Fig.~\ref{fig:Domain}).

%  \flashleft
\begin{figure}[!ht]
    % \centering
    %\flashleft
    \hspace{-0.7cm}
    \begin{minipage}[t]{0.45\textwidth}
        \begin{tikzpicture}[scale=1.5, z={(-.707,-.5)}]
            \draw (3.6,0,0) -- (0,0,0) -- (0,4,0)--(6,4,0)--(3.6,0,0);
            %\draw (3.6,0,0) -- (3.6,0,-1) -- (6,4,-1) -- (6,4,0) -- cycle;
            %\draw (6,4,0) -- (0,4,0) -- (0,4,-1) -- (6,4,-1);
            \node[anchor=south west,inner sep=0] at (0,0) {\includegraphics[width=1.13\textwidth]{Figs/Temp.png}};
            \draw (-0.5,2,0) node{};
            \draw (4.1,-.2,-1.5) node{};
            \draw [white] (3,3.8,0) node{$\rho_l = 1040$};
            \draw (1.8,2.1) node{$\rho_p = 1080$};
            \draw (3.1,2.1) node{$\rho_l = 1060$};
            \draw (2,0.2,0) node{$\rho_l = 1080$};
            \draw[<->] (-0.1,0) --node[above,rotate=90] {$H$} (-0.1,4);
            \draw[<->] (0,4.1,0) --node[above,] {$L_1$} (6,4.1,0);
            \draw[<->] (0,-0.1,0) --node[below,] {$L_2$} (3.7,-0.1,0);
            \draw [white, thick] (5.5,4) arc [start angle=180, end angle=240, radius=0.5cm]
        node [left] {$\alpha$};
            \draw[thick,->] (4.5,0.2,0) -- (5.5,0.2,0) node[anchor=north east]{$x$};
            \draw[thick,->] (4.5,0.2,0) -- (4.5,1+0.2,0) node[anchor=north west]{$z$};
            \draw[white,thick,->] (0,4.05,0) -- (1.5,2.8,0) ;
            \draw[thick,dotted] (0,2,0)--(4.8,2,0);
            \draw [white, thick] (0.0,3.5) arc [start angle=270, end angle=270+53, radius=0.5cm] node [below] {$\theta$}; 
        \end{tikzpicture}
    \subcaption{Initial distribution of density. $\rho_l$ -- density of liquid, $\rho_l$ -- density of particles, $\theta = \arcsin \frac{\omega}{N}$, where $\omega$ is frequency of wavemaker and $N$ is the buoyancy frequency.}
    \label{fig:domainup}
        \end{minipage}~~~~%
% \end{figure}
% \begin{figure}[b]
\hspace{0.5cm}
\begin{minipage}[t]{0.45\textwidth}
        \begin{tikzpicture}[scale=1.45]
            \draw [decorate,decoration={brace,amplitude=10pt},xshift=-4pt,yshift=0pt] (-0.2,0.0) -- (-0.2,4.1) node[rotate=90,above] [black,midway,xshift=-0,yshift=10] { Wavemaker};
            \node[anchor=south west,inner sep=0] at (0,0) {\includegraphics[width=1.13\textwidth]{Figs/Dom.png}};
%     \node[anchor=south west,inner sep=0] at (0,0) {\includegraphics[width=0.5265\textwidth]{Figs/Dom.png}};
            \draw[thick,<->] (-0.1,0) --node[above,rotate=90]{$H$} (-0.1,4.1);
            \draw[thick,<->] (0,-0.1) --node[below          ]{$L_2$} (3.8,-0.1);
            \draw[thick,<->] (0,4.2) --node[above          ]{$L_1$} (6.2,4.2);
            \draw [black, thick] (5.75,4.1) arc [start angle=180, end angle=270, radius=0.25cm];
        \end{tikzpicture}
    \subcaption{Ray-tracing of a internal wave beam subject to the dispersion relation~(\ref{eq:DispRel}).}
    \label{fig:Domain}
    \end{minipage}
    \caption{Scheme of computational domain for 2D attractor of internal waves. Initial distribution of density (a), and shape of attractor in ideal fluid (b).}
\end{figure}

%\begin{figure}[!ht]
%    \centering
%        \begin{tikzpicture}[scale=1.5, z={(-.707,-.5)}]
%            \draw (0,4-0,0) -- (6,4-0,0) -- (4,4-4,0)--(0,4-4,0) --cycle;
%            \draw (0,4-0,0)     -- (6,4-0,0)   -- (4,4-4,0)   -- (0,4-4,0)    -- cycle;
%            \draw[style = dashed,red] (2.7,4-4,0)   -- (0,4-1.8,0) -- (2.2,4-0,0) -- (4.97,4-2.1,0)-- cycle;
%            \draw[<->] (-0.1,4-0) --node[above,rotate=90] {$H$} (-0.1,4-4);
%            \draw[<->] (0,4+0.1,0) --node[above,] {$L_2$ } (6,4+0.1,0);
%            \draw[<->] (0,4-4.1,0) --node[below,] {$L_1$ } (4,4-4.1,0);
%            \draw[thick,->] (4.95,4-4,0) -- (5.95,4-4,0) node[anchor=north east]{$x$};
%            \draw[thick,->] (4.95,4-4,0) -- (4.95,4-3,0) node[anchor=north west]{$z$};
%            \draw[thick,->] (5.5,4-2,0) -- (5.5,4-3,0) node[anchor=west]{$\vec{g}$};
%        \end{tikzpicture}
%    \caption{Scratch of the domain and attraction area}
%    \label{fig:dominleft}
%\end{figure}

For the first time the phenomenon of wave attractors was described by Leo Maas \cite{Maas1995,1997MaasBenielliSommeriaLam} who studied the convergence of wave beams in different geometries.

First attempts of numerical study~\cite{2008GrisouardStaquetPairaud} of internal wave attractors successfully reproduced qualitatively the 2D structure of internal wave attractors in trapezoidal domains, though quantitatively the velocity amplitude was too high  as compared with the experiment. 3D numerical simulations~\cite{2016BrouzetSibgatullinScolanErmanyukDauxois,2016BrouzetErmanyukJoubaudSibgatullinDauxois} resolved this discrepancy through estimation of dissipation and boundary layers effects.

In this paper we apply the before-mentioned methods for the results of numerical experiments, and discuss the methods, which were not yet widely used for analysis of internal wave attractors.

% \section{Problem statement and solution methods}
\section{Numerical setup}

Laboratory and numerical study of linear and nonlinear dynamics of wave attractors in three-dimensional setup showed a good agreement in terms of major hydrodynamic properties both in laminar and turbulent regimes \cite{2016BrouzetSibgatullinScolanErmanyukDauxois,2016BrouzetErmanyukJoubaudSibgatullinDauxois}. High order spectral element approach, which was used in these simulations is still hard to adopt for multi-phase flows. In this study we consider the processes of sedimentation, and numerical simulation with the help of quasihydrodynamic  approach\cite{ElizarBook,SherBook}.

% For the stratified fluid dynamic simulation quasihydrodynamic equations were used \cite{ElizarBook}: 

Equations for mass, momentum, and salinity transport and diffusion in quasihydrodynamic approach can be written in the following form \cite{ElizarBook}:

 \begin{equation}
     \nabla \cdot \left (\vec U - \vec W \right ) = 0,
     \label{eq:cont}
 \end{equation}
 \begin{equation}
     \frac{\partial \vec U}{\partial t}  + \nabla \cdot \left ( (\vec U - \vec W)\otimes \vec U  \right )
     -
     \nabla \cdot \nu \left ( \nabla \vec U + (\nabla \vec U)^T \right ) - \nabla \cdot \left  (   \vec U \otimes \vec W \right ) 
      = - \frac{1}{\rho_m} \nabla \hat p + \vec F,
      \label{eq:mom}
 \end{equation}
 \begin{equation}
     \frac{\partial s}{\partial t} + \nabla \cdot \left ( (\vec U - \vec W)s \right )
     - \nabla \cdot \frac{\nu}{Sc} \left ( \nabla s \right )=0,
     \label{eq:tr}
 \end{equation}
\noindent
where  
%\noindent
reduced pressure $\hat p = p - p_0$, restoring force $\vec{F}=\beta \vec{g} \hat s$, $\hat s = s(x, y, z, t) - s(x, y, z, 0)$,
$\displaystyle \vec W = \tau \left ( \vec U \cdot \nabla \vec U + \frac{1}{\rho_m} \nabla \hat p - \vec F  \right )$.

Geometrically the computational domain is a rectangular trapeze, with the wavemaker located at the left wall, and inclined right wall, as in the experiments~\cite{2013ScolanErmanyukDauxois,2016BrouzetSibgatullinScolanErmanyukDauxois}.

% Velocity boundary conditions on wavemaker depends on problem being solved.

Boundary conditions at the fixed walls:

\begin{equation}\label{eq:qhd_walls}
        \vec{U} = 0, \,\,\, \frac{\partial \tilde p}{ \partial \vec{n}} = \rho_0 \vec n \cdot \left ( -\vec U_b \cdot \nabla \vec U + \vec F \right), \,\,\, \frac{\partial s}{ \partial \vec{n}} = 0.
\end{equation}

The initial conditions for a passive scalar are chosen so that the buoyancy frequency is equal to 1

\begin{equation}
    N(z) = \sqrt{- \frac{g}{\rho_m}\cdot\frac{d \rho(z)}{dz}} = 1,
\end{equation}
\noindent
where $\rho(z) = \rho_m(1+\beta s)$, $\rho_m$ is minimal density,  and $\beta$ -- coefficient of salinity contraction. 

Equations (\ref{eq:cont}--\ref{eq:tr}) were approximated with the help of finite volume method and open source code OpenFOAM v2012 with QHDFoam solver \cite{QGDFOam}

\section{Post-processing and data analysis}

The results of numerical simulation were approximated at a regular rectangular grid with the help of openFoam functionObject and processed by python scripts. Hydrodynamic fields were visualised with the help of open-source, multi-platform data analysis and visualization platform \emph{paraview} \cite{paraview}. 
% Scratches and schemes 

% \subsection{Two-dimensional simulations}

Parameters of the simulation are chosen to be close to the corresponding laboratory experiments.
As follows from the theoretical predictions and laboratory experiments, it is expected that internal waves energy will concentrate in the zone of attraction (Fig.~\ref{fig:Domain}). 

Geometry of the tank filled with salty water is given by: $H = 0.4 \; m$, $L_1=0.6 \; m$, $L_2 = 0.39 \; m$.
The buoyancy frequency $N=1$.

Boundary conditions at the left wall with the wavemaker read:

\begin{equation}
    U_x = a\cdot cos\left(\frac{\pi \cdot z}{H}\right)\cdot \omega_0 \cdot  sin(\omega_0 t),
    \label{eq:wmc}
\end{equation}
where $a = 0.003 \; m$, $\omega_0 = 0.63 \; s^{-1}$.

In the figures below we will use wavemaker frequency $\displaystyle f_0=\frac{\omega_0}{2\pi}$ and corresponding period $T_0=f_0^{-1}$. 

% Dynamic of velocity fluids field over time can be spit in two smooth parts.

% Part of attractor formation (Fig. \ref{fig:LamAttr}). Internal waves reflect from slope wall and aim to attractor with increasing own aplitude.

% Part of attractor destruction (Fig. \ref{fig:turbAttr}). Gradually amplitude grows so much that the waves begin to overturn and generate child waves.

\begin{figure}[h!]
\centering
    \begin{minipage}[t]{0.45\textwidth}
        \centering
            \animategraphics[autoplay,loop,scale=0.21]{10}{Figs/2DAttractorForm/300x200a03w0623.0}{180}{200}
            \subcaption[fir]{Internal waves attractor formation $t=0\,s$ to $t=80\,s$}
        \label{fig:LamAttr}
    \end{minipage}
    \begin{minipage}[t]{0.45\textwidth}
        \centering
            \animategraphics[autoplay,loop,scale=0.21]{10}{Figs/2DAttractorDest/300x200a03w0623.7}{160}{199}

            \subcaption[sec]{Internal waves turbulence $t=3590 \,s$ to $t=3600$}
        \label{fig:turbAttr}
    \end{minipage}
    \caption{Horizontal component of velocity, a) after formation of internal wave attractor with finite width, b) instability development and transition to fully turbulent motion.}
    \label{fig:2dAttr}
\end{figure}

Visualisation of velocity field was provided by open-source software paraview  \cite{paraview}. In figure \ref{fig:2dAttr} the horizontal component is shown. 
About 20 first wavemaker periods attractor is being formed, next the attractor holds the shape during about 30 periods, after 50 periods the perturbation are grown enough to initiate the cascade of the triadic instabilities. 
 
%to check
As the one of main method of flow motions analysis time-frequency diagram which shows dynamic of spectrum is considered (Fig. \ref{fig:tfd11}-\ref{fig:tfdX}). It is defined by the Fourier-transform made with a sliding time window. Hence, a vertical slice of the diagram is the spectrum taken over the neighbourhood of time moment $t$, so one can trace the evolution of the spectrum. At the start of the wavemaker the fluid has one clear frequency, but then the attractor becomes prone to hydrodynamic instabilities and the generation of the cascade of  secondary waves.
%, contamination of the spectrum, that appears as a noising background on the diagram. 

% \begin{figure}
% \begin{multicols}{2}
%     \centering
%     \includegraphics[width=0.55\textwidth]{Figs/TFBG14.png}
%     \caption{Time-frequency diagram of superharmonics presence ($a=1.4\;mm$)}
%     \label{fig:tfd11}
%     \hfill
%      \includegraphics[width=0.55\textwidth]{Figs/TFD_pos34693_nperseg400_to0T.png}
%     \caption{Time-frequency diagram of attractor destruction ($a=3.0\;mm$)}
%     \label{fig:tfdX}
% \end{multicols}
% \end{figure}


\begin{figure}
\centering
    \begin{minipage}[t]{0.45\textwidth}
        \centering
        \includegraphics[width=1.15\textwidth]{Figs/TFBG14.png}
        \subcaption[fir]{Time-frequency diagram of superharmonics presence ($a=1.4\;mm$)}
        \label{fig:tfd11}
    \end{minipage}
    \begin{minipage}[t]{0.45\textwidth}
        \centering
        \includegraphics[width=1.15\textwidth]{Figs/TFD_pos34693_nperseg400_to0T.png}
        \subcaption[sec]{Time-frequency diagram of attractor destruction ($a=3.0\;mm$)}
        \label{fig:tfdX}
    \end{minipage}
    \caption{Time-frequency diagrams}
\end{figure}


Figures \ref{fig:tfd11}-\ref{fig:tfdX} illustrate dependence of time-frequency diagrams on the forcing amplitude 
(log-scale of amplitude, normalized by maximum value).
% (colorbar shows the ratio of $\lg\left(\frac{TF}{\text{max} (TF)}\right)$ in order to normalize the maximum to 0).  
If the amplitude is not high enough, spectrum almost doesn't evolve and has discrete peaks; with the increase of the amplitude the spectrum combines discrete peaks and continuous background and fluctuates with the time, which indicates the transition to the turbulent regime.

% \newpage
\subsection{Sedimentation}

One of the important problems with applications in oceanology is the sedimentation of suspended particles and their interaction with the internal waves. To describe qualitatively the influence of wave attractors on sedimentation of suspended particles we have made visualization for laminar and turbulent regimes of wave attractors. Initial density profile in  the tank is linear.
% from $1040\;\frac{kg}{m^3}$ to $1080\;\frac{kg}{m^3}$, 
% At the middle horizontal line of the reservoir spherical particles with density $\rho_p = 1080;\frac{kg}{m^3}$ and were uniformly distributed.  
At the initial state spherical particles with diameter $d=0.001 \; m$ were distributed uniformly over the horizontal middle line (Fig. \ref{fig:domainup}). 
% Particles are floating with the fluid, but 
The model does not account for feedback from the particles on the flow. 
%In this simulation there is no feedback from the particles on the flow. 
Such a setup showed drastic differences in the sedimentation properties for laminar and turbulent wave attractors.

\begin{figure}[h!]
\centering
    \begin{minipage}[t]{0.45\textwidth}
        \centering
        \animategraphics[autoplay,loop,scale=0.2]{10}{Figs/AttrctorSlipLaminarSediment/SlipLam.0}{000}{159}
%        \subcaption[fir]{Sedimentation from $t=0\,s$ to $t=80\,s$}

        \subcaption[fir]{Sedimentation from $t=0\,s$ to $t=80\,s$}
        \label{fig:turbSed2Da}
    \end{minipage}
    \begin{minipage}[t]{0.45\textwidth}
        \centering
        \animategraphics[autoplay,loop,scale=0.2]{10}{Figs/AttrctorSlipLaminarMid/SlipLam.07}{40}{79}

        \subcaption[sec]{Redistribution from $t=3590 \,s$ to $t=3600$}
        \label{fig:turbSed2Db}
    \end{minipage}
    \caption{Sedimentation in presence of stable (``laminar'') wave attractor.}
\end{figure}
%\vspace {-0.5cm}
\begin{figure}[h!]
\centering
    \begin{minipage}[t]{0.45\textwidth}
        \centering
    \animategraphics[autoplay,loop,scale=0.2]{10}{Figs/2DSedTurbBegin/AnimationRho1080C0r01mmslip.0}{001}{159}

        \subcaption[fir]{Sedimentation from $t=0\,s$ to $t=80\,s$}
        \label{fig:turbSed2Da}
    \end{minipage}
    \begin{minipage}[t]{0.45\textwidth}
        \centering
    \animategraphics[autoplay,loop,scale=0.2]{10}{Figs/2DSedTurbEnd/AnimationRho1080C0r01mmslip.7}{013}{052}

        \subcaption[sec]{Redistribution from $t=3590 \,s$ to $t=3600$}
        \label{fig:turbSed2Db}
    \end{minipage}
%    \caption{Sedimentation in reservoir with turbulent regime of internal waves attractor}
    \caption{Sedimentation in presence of turbulent wave attractor.}
    \label{fig:turbSed}
\end{figure}

Two questions arise: a) redistribution of the particles in the bulk flow before the particles approach the bottom, b) redistribution of the particles near the bottom.

% The simulation is of interest for two reasons: sedimentation process before particles touch the bottom and distribution process after sedimentation. 
Numerical experiments were carried out for two different regimes, corresponding to wavemaker (\ref{eq:wmc})  amplitudes $a_1=0.0005\;m$ -- regime of stable attractor without wave instabilities, and $a_2=0.003\;m$ -- fully turbulent regime. 

Numerical experiment have shown that particles experience oscillation while falling, and in laminar regime they continue to oscillate near the bottom. For the turbulent regime the long-time behaviour of the particles near the bottom is completely different: all the particles are attracted by two points at the bottom during the period $1000-2000\ s$ (Fig. \ref{fig:turbSed}).

% Changing of experiment condition for amplitude of wavemaker to $A_1$ leads to the fact that attractor do not redistribute particles after sedimentation. Also, the attractor is too weak to significantly influence to process of sedimentation.

%%%%%%%%%%%%%%%


\subsection{Three-dimensional wavemaker}

Three-dimensional setup may result in additional focusing along the transverse (spanwize) direction\cite{2018PilletErmanyukMaasSibgatDauxois}.
Despite the fact that attractor may be located at only part of the left wall, the visualisation of numerical simulation in figures 7 (a,b) shows almost perfect two-dimensional structure. 

% Focusing of internal waves occurs in three-dimensional mode before almost 2D structure is achieved. Waves propagate across three directions and after numerous reflections concentrates about attractors surface, which can be obtained from 2D mode by stretching along the direction perpendicular to the plane of the tank. 

% This effect is especially remarkable if the wavemaker takes up only half of the left wall (Fig. \ref{fig:3DLocSctarch}). For saving total energy incoming to the system one need to increase wavemaker amplitude for (\ref{eq:wmc}) $A=0.006\ m$. Size in $X-Z$-plane same as 2D case, 3D reservoir has $0.08\ m$ along $Y$ direction. Walls has a slip condition for velocity and pressure.

\begin{figure}
    \centering
      \begin{tikzpicture}[scale=1.5, every node/.style={scale=1.2}]
        \filldraw[top color=red, bottom color=blue] (0,0,0)--(0,0,0.4)--(0,4,0.4)--(0,4,0)--cycle;
%        
        \draw (0,4,0.0) -- (6,4,0.0)-- (3.9,0,0.0);
        \draw (3.9,0,0.8) -- (0,0,0.8) -- (0,4,0.8) -- (6,4,0.8) -- cycle;
%
        \draw (0,4,0) -- (0,4,0.8);
%
        \draw (3.9,0,0) -- (3.9,0,0.8);
        \draw (6,4,0) -- (6,4,0.8);

        % \draw[style=dashed, color=gray] (6,0,-3) -- (0,0,-3) -- (0,4,-3);
        \draw[style=dashed, color=gray] (0,0,0) -- (0,0,1.5);
        \draw[style=dashed, color=gray] (3.9,0,0.8) -- (3.9,0,1.5);
%
        \draw[style=dashed, color=gray] (0,4,0) -- (0,4,-1.);
        \draw[style=dashed, color=gray] (6,4,0.8) -- (6,4,-1.);
%
        \draw[stealth-stealth] (0,4,-0.5)--node[above]{$L_1$}(6,4,-0.5);
%
        \draw[stealth-stealth] (0,0,1.3)--node[below]{$L_2$}(3.9,0,1.3);
%
        \draw[style=dashed, color=gray] (0,0,0) -- (3.9,0,0);
%
        \draw[style=dashed, color=gray] (6,4,0.8) -- (7,4,0.8);
        \draw[style=dashed, color=gray] (6,4,0.0) -- (7,4,0.0);
        \draw[stealth-stealth] (7,4,0.8)--node[below,rotate=50]{K}(7,4,0.0);
        % \draw (-0.5,2,0) node{};
        % \draw (4.1,-.2,-3.5) node{};
        % \draw (3.5,5,0) node{};
        \draw[stealth-stealth] (-0.2,0,0.8) --node[above,rotate=90] {$H$} (-0.2,4,0.8);
        \draw[style=dashed, color=gray] (-0.3,0,0.8) -- (0.0,0,0.8);
        \draw[style=dashed, color=gray] (-0.3,4,0.8) -- (0.0,4,0.8);
        % \draw[<->] (0,4.1,-3) --node[above,] {$L_1 = 0.369$ m} (4,4.1,-3);
        % \draw[<->] (6.05,-0.05,0) --node[below,rotate=37] {$W = 0.3$ m} (6.05,-0.05,-3);
        % \draw[<->] (0,-0.1,0) --node[below,] {$L_2 = 0.6$ m} (6,-0.1,0);
        % \draw [black, thick] (3.75,4) arc [start angle=180, end angle=300, radius=0.25cm]
        % node [anchor = south west] {$120^\circ$};
        \draw[thick,->] (4.8,0.05,0) -- (5.8,0.05,0) node[anchor=north east]{$x$};
        \draw[thick,->] (4.8,0.05,0) -- (4.8,1.05,0) node[anchor=east]{$z$};
        \draw[thick,->] (4.8,0.05,0) -- (4.8,0.05,-1) node[anchor=south]{$y$};
        \draw[thick,->] (6,3,0) -- (6,1,0) node[anchor=west]{$\vec{g}$};
      \end{tikzpicture}
      
    \caption{Illustration of 3D domain with local wavemaker in numerical experiment $H=0.4\ m$, $L_1 = 0.6\ m$, $L_2 = 0.39\ m$, $K=0.08 \ m$.}
    \label{fig:3DLocSctarch}
\end{figure}

With addition third dimension simulation problem complicated significantly. There are difficulties in visualizing a three-dimensional velocity field. For showing destruction of internal waves attractor paraview filter 'threshold' was used. It allows to see values from gap and detect areas with intensive motion (Fig. \ref{fig:3DLocTurbRes}). Emptiness is proposed to fill in salinity field with interprocess patches (glares). And finally with 'slice' filter select closest to wavemaker area for visualisation of wavemaker motion. 



\begin{figure}[!ht]
\centering
    \begin{minipage}{0.45\textwidth}
        \centering
            \animategraphics[autoplay,loop,scale=0.11]{10}{Figs/3DPartTurbAttr6/3dTurbPartA6.00}{01}{40}

            \subcaption[f]{Stable phase of internal waves attractor for local wavemaker from $t=150\,s$ to $t=170\,s$}
        \label{fig:3DTurbBegin}
    \end{minipage}
    \begin{minipage}{0.45\textwidth}
        \centering
            \animategraphics[autoplay,loop,scale=0.11]{10}{Figs/3DPartTurbAttr6/3dTurbPartA6.02}{59}{99}

            \subcaption[s]{Turbulisation of internal waves attrctor with local wavemaker from $t=280\,s$ to $t=300\,s$}
        \label{fig:3DTurbEnd}
    \end{minipage}
    \caption{Life cycle of internal waves attractor from stability to chaos. Color visualisation of velocity horizontal component. Flare in reservoir is a interprocessors interfaces painted in salinity colorbar where $1$ is a blue and $2$ is a red. Wall with wave maker has a different color scale: from $-0.001\,m/s$ to $0.001\,m/s$.}
    \label{fig:3DLocTurbRes}
\end{figure}

    Visual analysis of simulation show internal waves attractors destruction procedure. The attractor is formed in the period from 0 to 10 oscillation periods, then from 10 to 30 oscillation periods stable and attractor structure is observed and finally from 30 to 300 and belong difference of velocities become too big that waves overturn and incoming energy produce secondary waves.

\newpage
\subsection{Vortices visualisation}

Vortices identification is a relatively new way to analyze the structure of turbulent motions. For this purpose several approach have been developed.

\begin{figure}[H]
\begin{multicols}{2}
    \centering
    \includegraphics[width=0.5\textwidth]{Figs/VortexRot.png}
    \caption{Vorticity}
    \label{fig:vorticity}
    \hfill
    \includegraphics[width=0.5\textwidth]{Figs/VortexDelta.png}
    \caption{$\Delta$-method}
    \label{fig:Delta}
\end{multicols}
\end{figure}

The simplest method is the analysis of vorticity \cite{vortex} (Fig.~\ref{fig:vorticity}): $$\omega=\textrm{rot}\,\vec{v}$$ and stream function.
This approach however can result in ambiguous conclusions. For instance, laminar plane Couette has a constant finite vorticity everywhere, and still, there is no structure, which could be identified as ``vortices'' or ``swirls'' in this type of flow.

\begin{figure}[H]
\begin{multicols}{2}
    \centering
    \includegraphics[width=0.5\textwidth]{Figs/VortexQ.png}
    \caption{$Q$-method}
    \label{fig:Q}
    \hfill
    \includegraphics[width=0.5\textwidth]{Figs/VortexLambdaCi.png}
    \caption{$\lambda_{ci}$-method}
    \label{fig:Lci}
    \end{multicols}
\end{figure}

Another approach is a so-called $\Delta$-method \cite{vortex}. Let us consider eigenvalue problem for velocity gradient tensor $A=\nabla\otimes  \vec v$:

% \begin{equation}
%  A:= \nabla\otimes  \vec v =
% \begin{pmatrix}
%  \frac{\partial v_x}{\partial x}  & \frac{\partial v_x}{\partial y} \\ \frac{\partial v_y}{\partial x} & \frac{\partial v_y}{\partial y} 
% \end{pmatrix}
% \end{equation}

\begin{equation}
  \text{det} \,( A -\lambda \mathbb E)=0
\end{equation}  

In three-dimensional setup we have the following characteristic equation: $$\lambda^3+P\lambda^2+Q\lambda+R,$$ 
\noindent where discriminant $D=-108\left( \frac{Q^2}{4} +\frac{P^3}{27}\right)$. 



Let's define $$\Delta=\left(\frac{Q}{2}\right)^2 +\left(\frac{P}{3}\right)^3.$$

Then 
$ \Delta > 0  \Leftrightarrow  D < 0$ corresponds to the presence of two complex-conjugated roots, and can be interpreted as spiral-shaped stream tubes.

In two-dimensional setup, the equation has lower order:
$$\lambda^2+P\lambda+Q=0,$$

\noindent
where discriminant $D=P^2-4Q$. Thus in 2D we may take $-D\equiv 4Q-P^2$ as $\Delta$ (Fig.~\ref{fig:Delta}).

This method requires more computational resources than vorticity, but the vortices as rotating objects can be readily identified. It can be improved using $Q$-method \cite{vortex}\cite{Hussain}. Let's define:

\begin{figure}
\begin{multicols}{2}
    \centering
    \includegraphics[width=0.5\textwidth]{Figs/VortexLambda2.png}
    \caption{$\lambda_2$-method}
    \label{fig:L2}
    \hfill
    \includegraphics[width=0.5\textwidth]{Figs/VortexLui.png}
    \caption{Liutex-method}
    \label{fig:Liu}
    \end{multicols}
\end{figure}

\begin{equation}
 S=\frac{1}{2}\left(A +A\,^T \right) ,\quad \Omega=\frac{1}{2} \left( A - A\,^T\right) 
  \label{NSdim}
\end{equation}
Then
\begin{equation}
  Q=\frac{1}{2} \left( ||\Omega||^2_F-||S||^2_F \right) in ideal fluid 
  \label{eqn:Q}
 \end{equation}
The norm in (\ref{eqn:Q}) is understood in sense of Frobenius norm, i.e. $||A||^2_F=\sum \limits_{i=1}^n \sum \limits_{j=1}^n a_{ij}^2$, where $n$ is problem dimension.

The vortex regions are those with $Q>0$. It is a harder restriction then that for the $\Delta$, so $Q$-identified vortex regions a priori included in $\Delta$-identified ones (Fig.~\ref{fig:Q}).

For the fourth method let's calculate eigenvalues of $A$. For the vortex presence it's necessary for two complex-conjugation values to exist. Their imaginary parts $\pm\lambda_{ci}$ that can be considered as the identification method itself \cite{vortex}, fig.~\ref{fig:Lci}.

If we consider eigenvalues of $S^2+\Omega^2$~matrix, we obtain $\lambda_2$-method. They cannot be complex, so for the vortex two negative ones are implied. Let's sort them in descending order and consider the second one as a criterion\cite{vortex,Hussain}. For the visibility we will display it with the opposite sign (hence it's negative), Fig.~\ref{fig:L2}.

The most recent approach is Luitex-method \cite{vortex}.

Let $\vec r \longleftrightarrow \lambda_r$~-- eigenvector corresponding to real eigenvalue $\lambda_r, ||\vec r||=1$. Then Luitex-vector is introduced as the one with the direction of $\vec r$ and value
 $$R=(\vec \omega,\vec r)-\sqrt{(\vec \omega,\vec r)^2-4\lambda_{ci}^2},$$\noindent where $\vec \omega$ is vorticity. Wherein it's supposed that $(\vec \omega,\vec r)>0$.
 
 In 2-dimensional problem Luitex-vector must be re-defined. Obviously, vorticity is orthogonal to the domain plane as well as $\vec r$. Hence we consider only value
 
 $$R=|\omega|- \sqrt{\omega^2-4\lambda_{ci}^2},$$\noindent where $\omega=\frac{\partial v_y}{\partial v_x}-\frac{\partial v_x}{\partial v_y}$~-- two-dimensional vorticity (Fig.~\ref{fig:Liu}).

Though the described methods are aimed to visualize vortex regions, they can be applied for the estimation of the properties of the attractors.

\section{Conclusions}

As a result of direct numerical simulation of internal wave turbulence, one gets a large volume of data. It is now possible to get high-resolution data, and resolve the small-scale flows. At the same time, it requires more computational resources to generate and process it. Raw data is difficult to interpret without special software. In this paper we consider some open-source software for flow analysis and visualization. The open-source software complies with the principles of scientific credibility, ability to modify and reuse the code, and source code transparency.
% and high requirements for user competencies.

To obtain hydrodynamic fields OpenFOAM was used, but raw data takes up a lot of space on hard drives and generate a lot of files especially in simulations with particles. Sometimes it can be critical. At this paper functionObject 'vtkWrite' was used to resolve this problem. For simulation with particles 'cloudWrite' function object was used which generate much fewer files then raw pure OpenFOAM output. Additionally, data was collected at key nodes for Fourier analysis and construction of time-frequency diagrams with 'postProcWrite' functionObject.

Problem of internal gravity waves attractor simulation is complicated by the fact that it is required to simulate significant periods of model time. 24 processors for $300 \times 200$ computational domain needs for simulation of 2D problem. Simulation of a 2D problem lasted for 5 days on Unihub cluster \cite{Unihub2014}. 3D problem required more detailed mesh additionally to the 3-d dimension. 196 processors for $375 \times 250 \times 16$ domain. Simulation of 450 seconds lasted for 7 days on The Lobachevsky supercomputer.

Processing of simulation results required many different instruments for calculations and visualisation.
Fourier analysis was provided with python package scipy \cite{scipy}, and drawing with matplotlib \cite{matplotlib}. Visualisation was obtained by paraview \cite{paraview}. Scripts for vortices visualisation was developed with python \cite{python}. 

\section{Acknowledgments}

The research was supported by RSF (project No. 19-11-00169).
Authors thank Elizarova T. G. for continuous support.

\bibliographystyle{ugost2008}
\bibliography{rpz,internalwaves}

\end{document}
